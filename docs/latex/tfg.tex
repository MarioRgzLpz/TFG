% !TEX program = pdflatex
% !TEX encoding = UTF-8 Unicode

% Plantilla, basada en la clase `scrbook` del paquete KOMA-script,  para la elaboración de un TFG siguiendo las directrices del la comisión del Grado en Matemáticas de la Universidad de Granada.

% Francisco Torralbo Torralbo

\documentclass[print, color]{ugrTFG}

% VERSIÓN ELECTRÓNICA PARA TABLETA
% Cambiando la opción "print" por "tablet" generaremos un pdf adaptado para leerlo en tabletas de 9 pulgadas.

% -------------------------------------------------------------------
% INFORMACIÓN DEL TFG Y EL AUTOR
% -------------------------------------------------------------------

\newcommand{\miTitulo}{Criptografía basada en retículos \xspace}
\newcommand{\miNombre}{Mario Rodríguez López\xspace}
\newcommand{\miGrado}{Doble Grado en Matemáticas e Ingeniería Informática}
\newcommand{\miFacultad}{Facultad de Ciencias y Escuela Técnica Superior de Ingenierías Informática y Telecomunicación}
\newcommand{\miUniversidad}{Universidad de Granada}

% Añadir tantos tutores como sea necesario separando cada uno de ellos mediante el comando `\medskip` y una línea en blanco
\newcommand{\miTutor}{
  Francisco Javier Lobillo Borrero \\ \emph{Departamento de Álgebra} 

  % Añadir tantos tutores como sea necesario. 

  \medskip
  Nombre del tutor 2 \\ \emph{Departamento del tutor 2}
}
\newcommand{\miCurso}{2024-2025\xspace}

\hypersetup{
	pdftitle={\miTitulo},
	pdfauthor={\textcopyright\ \miNombre, \miFacultad, \miUniversidad}
}

\begin{document}

\maketitle

% -------------------------------------------------------------------
% FRONTMATTER
% -------------------------------------------------------------------
\frontmatter % Desactiva la numeración de capítulos y usa numeración romana para las páginas

% !TeX root = ../tfg.tex
% !TeX encoding = utf8
%
%*******************************************************
% Declaración de originalidad
%*******************************************************

\thispagestyle{empty}

\hfill\vfill

\textsc{Declaración de originalidad}\\\bigskip

D./Dña. \miNombre \\\medskip

Declaro explícitamente que el trabajo presentado como Trabajo de Fin de Grado (TFG), correspondiente al curso académico \miCurso, es original, entendido esto en el sentido de que no he utilizado para la elaboración del trabajo fuentes sin citarlas debidamente.
\medskip

En Granada a \today 
\vspace{3cm}
\begin{center} 
Fdo: \miNombre 

\end{center}

\vfill

\cleardoublepage
\endinput
   
% !TeX root = ../tfg.tex
% !TeX encoding = utf8

%*******************************************************
% Dedication
%*******************************************************
\thispagestyle{empty}
\phantomsection 
\pdfbookmark[1]{Dedicatoria}{Dedicatoria}

\hfill
\vfill

\begin{flushright}
\itshape
A mis padres, por apoyarme siempre
\end{flushright}

\vfill

\cleardoublepage
\endinput
                % Opcional
% !TeX root = ../tfg.tex
% !TeX encoding = utf8

%*******************************************************
% Table of Contents
%*******************************************************
\phantomsection
\pdfbookmark[0]{\contentsname}{toc}

\setcounter{tocdepth}{2} % <-- 2 includes up to subsections in the ToC
\setcounter{secnumdepth}{3} % <-- 3 numbers up to subsubsections

\tableofcontents 

%*******************************************************
% List of Figures and of the Tables
%*******************************************************

    % *******************************************************
    %  List of Figures
    % *******************************************************    
    \phantomsection 
    % \listoffigures

    %*******************************************************
    % List of Tables
    %*******************************************************
    \phantomsection 
    % \listoftables
    
    %*******************************************************
    % List of Listings
    % The package \usepackage{listings} is needed
    %*******************************************************      
	  % \phantomsection 
    % \renewcommand{\lstlistlistingname}{Listados de código}
    % \lstlistoflistings 

\cleardoublepage
            
% !TeX root = ../tfg.tex
% !TeX encoding = utf8

%*******************************************************
% Agradecimientos
%*******************************************************

\chapter{Agradecimientos}

Agradecimientos (opcional, ver archivo \texttt{preliminares/agradecimiento.tex}).

\cleardoublepage
\endinput
            % Opcional

% !TeX root = ../tfg.tex
% !TeX encoding = utf8
%
%*******************************************************
% Summary
%*******************************************************

\selectlanguage{english}
\chapter{Summary}

An english summary of the project (around 800 and 1500 words are recommended).

File: \texttt{preliminares/summary.tex}


% Al finalizar el resumen en inglés, volvemos a seleccionar el idioma español para el documento
\selectlanguage{spanish} 
\endinput
                    
% !TeX root = ../tfg.tex
% !TeX encoding = utf8
%
%*******************************************************
% Introducción
%*******************************************************

% \manualmark
% \markboth{\textsc{Introducción}}{\textsc{Introducción}} 

\chapter{Resúmen}

De acuerdo con la comisión de grado, el TFG debe incluir una introducción en la que se describan claramente los objetivos previstos inicialmente en la propuesta de TFG, indicando si han sido o no alcanzados, los antecedentes importantes para el desarrollo, los resultados obtenidos, en su caso y las principales fuentes consultadas.

Ver archivo \texttt{preliminares/introduccion.tex}

\endinput
               

% -------------------------------------------------------------------
% MAINMATTER
% -------------------------------------------------------------------
\mainmatter % activa la numeración de capítulos, resetea la numeración de las páginas y usa números arábigos

% !TeX root = ../tfg.tex
% !TeX encoding = utf8

\chapter*{Motivación}
Vivimos en una era donde la digitalización ha transformado todos los aspectos de nuestra vida cotidiana. Desde la forma en que nos comunicamos, hasta cómo almacenamos información y accedemos a servicios esenciales, la dependencia de las tecnologías digitales es innegable. Esta digitalización, aunque ofrece innumerables beneficios en términos de eficiencia y conveniencia, también presenta desafíos significativos en cuanto a la protección de la información y la privacidad. Aquí es donde entra en juego la criptografía.\\

La criptografía es esencial en nuestra vida diaria, aunque muchas veces pase desapercibida. Su importancia radica en la capacidad de proteger la información confidencial, garantizar la privacidad y asegurar la integridad de los datos en un mundo cada vez más digitalizado. Desde el uso de tarjetas de crédito y transacciones bancarias en línea hasta la comunicación a través de aplicaciones de mensajería y el almacenamiento de datos personales en la nube, la criptografía asegura que estos procesos sean seguros y que la información no caiga en manos equivocadas.\\

Con el avance imparable de la tecnología, nos enfrentamos a una amenaza que desafía la seguridad de los métodos criptográficos tradicionales: la computación cuántica. \\

Actualmente, los algoritmos criptográficos más utilizados, como RSA y ECC (Criptografía de Curva Elíptica), se basan en problemas matemáticos complejos, como la factorización de números enteros grandes y el logaritmo discreto, que son extremadamente difíciles de resolver con la computación clásica. Sin embargo, estos problemas pueden ser resueltos de manera eficiente por los ordenadores cuánticos utilizando el algoritmo de Shor.\\

El algoritmo de Shor, desarrollado por el matemático Peter Shor en 1994, es capaz de factorizar números enteros grandes y resolver problemas de logaritmos discretos en un tiempo significativamente menor que los algoritmos clásicos. Esto implica que los ordenadores cuánticos podrían romper la mayoría de los sistemas criptográficos actuales, exponiendo información confidencial y comprometiendo la seguridad de los datos.\\

En este contexto, surge la necesidad urgente de desarrollar y adoptar sistemas criptográficos que sean resistentes a los ataques de la computación cuántica. Aquí es donde entra en juego el criptosistema post-cuántico Kyber-Crystals, una de las propuestas más prometedoras en el ámbito de la criptografía post-cuántica y el que fue seleccionado como el nuevo estándar de criptografía por el National Institute of Standards and Technology (NIST) en el año 2022, un reconocimiento que subraya su importancia y robustez frente a las amenazas cuánticas.\\

Kyber-Crystals se basa en problemas matemáticos en retículos o redes, que son considerados intratables incluso para los ordenadores cuánticos. Este enfoque garantiza que los datos cifrados bajo este sistema permanezcan seguros, resistiendo tanto a los ataques tradicionales como a los potenciales ataques cuánticos. La adopción de Kyber-Crystals como estándar de criptografía por el NIST es un hito significativo, ya que refleja una confianza institucional en su capacidad para proteger la información en un futuro dominado por la tecnología cuántica.


\endinput
%--------------------------------------------------------------------
% FIN DEL CAPÍTULO. 
%--------------------------------------------------------------------

\part{Fundamento Teórico. Reticulos, MDLW \& Crystals-Kyber} % Dividir un TFG en partes OPCIONAL
% !TeX root = ../tfg.tex
% !TeX encoding = utf8

\chapter{Preliminares}

\section{Definición}

Un retículo en $\mathbb{R}^n$ es un subgrupo aditivo discreto de $\mathbb{R}^n$ definido como el conjunto de todas las combinaciones lineales enteras de n vectores linealmente independientes. Este conjunto de vectores se conoce como una base del retículo y no es único. Es decir un reticulo es el conjunto de todas las combinaciones lineales enteras de vectores de una base $B=\{b_1,b_2,\ldots,b_n\} \subset \mathbb{R}^n$.

\begin{equation}
    \mathcal{L} = \left\{ \sum_{i=1}^{n} a_i b_i \mid a_i \in \mathbb{Z} \right\}
\end{equation}


\endinput
%--------------------------------------------------------------------
% FIN DEL CAPÍTULO. 
%--------------------------------------------------------------------
% !TeX root = ../tfg.tex
% !TeX encoding = utf8

\chapter{Capítulo 2}

\section{Primera sección}

Este fichero \texttt{capitulo-ejemplo.tex} es una plantilla para añadir capítulos al \textsc{tfg}. Para ello, es necesario:
\begin{itemize}
  \item Crear una copia de este fichero \texttt{capitulo-ejemplo.tex} en la carpeta \texttt{capitulos} con un nombre apropiado (p.e. \texttt{capitulo01.tex}).
  \item Añadir el comando \texttt{$\backslash$input\{capitulos/capitulo01\}} en el fichero principal \texttt{tfg.tex} donde queremos que aparezca dicho capítulo.
\end{itemize}


\endinput
%--------------------------------------------------------------------
% FIN DEL CAPÍTULO. 
%--------------------------------------------------------------------
% !TeX root = ../tfg.tex
% !TeX encoding = utf8

\chapter{Capítulo 3}

\section{Primera sección}

Este fichero \texttt{capitulo-ejemplo.tex} es una plantilla para añadir capítulos al \textsc{tfg}. Para ello, es necesario:
\begin{itemize}
  \item Crear una copia de este fichero \texttt{capitulo-ejemplo.tex} en la carpeta \texttt{capitulos} con un nombre apropiado (p.e. \texttt{capitulo01.tex}).
  \item Añadir el comando \texttt{$\backslash$input\{capitulos/capitulo01\}} en el fichero principal \texttt{tfg.tex} donde queremos que aparezca dicho capítulo.
\end{itemize}


\endinput
%--------------------------------------------------------------------
% FIN DEL CAPÍTULO. 
%--------------------------------------------------------------------
% !TeX root = ../tfg.tex
% !TeX encoding = utf8

\chapter{Capítulo 4}

\section{Primera sección}

Este fichero \texttt{capitulo-ejemplo.tex} es una plantilla para añadir capítulos al \textsc{tfg}. Para ello, es necesario:
\begin{itemize}
  \item Crear una copia de este fichero \texttt{capitulo-ejemplo.tex} en la carpeta \texttt{capitulos} con un nombre apropiado (p.e. \texttt{capitulo01.tex}).
  \item Añadir el comando \texttt{$\backslash$input\{capitulos/capitulo01\}} en el fichero principal \texttt{tfg.tex} donde queremos que aparezca dicho capítulo.
\end{itemize}


\endinput
%--------------------------------------------------------------------
% FIN DEL CAPÍTULO. 
%--------------------------------------------------------------------
% !TeX root = ../tfg.tex
% !TeX encoding = utf8

\chapter{Crystals-Kyber funcionamiento y dureza.}

\section{Primera sección}

Este fichero \texttt{capitulo-ejemplo.tex} es una plantilla para añadir capítulos al \textsc{tfg}. Para ello, es necesario:
\begin{itemize}
  \item Crear una copia de este fichero \texttt{capitulo-ejemplo.tex} en la carpeta \texttt{capitulos} con un nombre apropiado (p.e. \texttt{capitulo01.tex}).
  \item Añadir el comando \texttt{$\backslash$input\{capitulos/capitulo01\}} en el fichero principal \texttt{tfg.tex} donde queremos que aparezca dicho capítulo.
\end{itemize}


\endinput
%--------------------------------------------------------------------
% FIN DEL CAPÍTULO. 
%--------------------------------------------------------------------
% !TeX root = ../tfg.tex
% !TeX encoding = utf8

\chapter{Posibles ataques a Crystals-kyber.}

\section{Primera sección}

Este fichero \texttt{capitulo-ejemplo.tex} es una plantilla para añadir capítulos al \textsc{tfg}. Para ello, es necesario:
\begin{itemize}
  \item Crear una copia de este fichero \texttt{capitulo-ejemplo.tex} en la carpeta \texttt{capitulos} con un nombre apropiado (p.e. \texttt{capitulo01.tex}).
  \item Añadir el comando \texttt{$\backslash$input\{capitulos/capitulo01\}} en el fichero principal \texttt{tfg.tex} donde queremos que aparezca dicho capítulo.
\end{itemize}


\endinput
%--------------------------------------------------------------------
% FIN DEL CAPÍTULO. 
%--------------------------------------------------------------------
% !TeX root = ../tfg.tex
% !TeX encoding = utf8

\chapter{Capítulo 7}

\section{Primera sección}

Este fichero \texttt{capitulo-ejemplo.tex} es una plantilla para añadir capítulos al \textsc{tfg}. Para ello, es necesario:
\begin{itemize}
  \item Crear una copia de este fichero \texttt{capitulo-ejemplo.tex} en la carpeta \texttt{capitulos} con un nombre apropiado (p.e. \texttt{capitulo01.tex}).
  \item Añadir el comando \texttt{$\backslash$input\{capitulos/capitulo01\}} en el fichero principal \texttt{tfg.tex} donde queremos que aparezca dicho capítulo.
\end{itemize}


\endinput
%--------------------------------------------------------------------
% FIN DEL CAPÍTULO. 
%--------------------------------------------------------------------

\part{Implementaciones realizadas. Metodologia usada para el desarrollo y codigo implementado}
% !TeX root = ../tfg.tex
% !TeX encoding = utf8

\chapter{KANBAN y toma de decisiones}

\section{Primera sección}

Este fichero \texttt{capitulo-ejemplo.tex} es una plantilla para añadir capítulos al \textsc{tfg}. Para ello, es necesario:
\begin{itemize}
  \item Crear una copia de este fichero \texttt{capitulo-ejemplo.tex} en la carpeta \texttt{capitulos} con un nombre apropiado (p.e. \texttt{capitulo01.tex}).
  \item Añadir el comando \texttt{$\backslash$input\{capitulos/capitulo01\}} en el fichero principal \texttt{tfg.tex} donde queremos que aparezca dicho capítulo.
\end{itemize}


\endinput
%--------------------------------------------------------------------
% FIN DEL CAPÍTULO. 
%--------------------------------------------------------------------
% !TeX root = ../tfg.tex
% !TeX encoding = utf8

\chapter{Implementación del criptosistema}

\section{Primera sección}

Este fichero \texttt{capitulo-ejemplo.tex} es una plantilla para añadir capítulos al \textsc{tfg}. Para ello, es necesario:
\begin{itemize}
  \item Crear una copia de este fichero \texttt{capitulo-ejemplo.tex} en la carpeta \texttt{capitulos} con un nombre apropiado (p.e. \texttt{capitulo01.tex}).
  \item Añadir el comando \texttt{$\backslash$input\{capitulos/capitulo01\}} en el fichero principal \texttt{tfg.tex} donde queremos que aparezca dicho capítulo.
\end{itemize}


\endinput
%--------------------------------------------------------------------
% FIN DEL CAPÍTULO. 
%--------------------------------------------------------------------
% !TeX root = ../tfg.tex
% !TeX encoding = utf8

\chapter{Ataque a crystals-kyber}

\section{Primera sección}

Este fichero \texttt{capitulo-ejemplo.tex} es una plantilla para añadir capítulos al \textsc{tfg}. Para ello, es necesario:
\begin{itemize}
  \item Crear una copia de este fichero \texttt{capitulo-ejemplo.tex} en la carpeta \texttt{capitulos} con un nombre apropiado (p.e. \texttt{capitulo01.tex}).
  \item Añadir el comando \texttt{$\backslash$input\{capitulos/capitulo01\}} en el fichero principal \texttt{tfg.tex} donde queremos que aparezca dicho capítulo.
\end{itemize}


\endinput
%--------------------------------------------------------------------
% FIN DEL CAPÍTULO. 
%--------------------------------------------------------------------

% -------------------------------------------------------------------
% APPENDIX: Opcional
% -------------------------------------------------------------------

\appendix % Reinicia la numeración de los capítulos y usa letras para numerarlos
\pdfbookmark[-1]{Apéndices}{appendix} % Alternativamente podemos agrupar los apéndices con un nuevo \part{Apéndices}

% !TeX root = ../tfg.tex
% !TeX encoding = utf8

\chapter{Ejemplo de apéndice}\label{ap:apendice1}

Los apéndices son opcionales.

Este fichero \texttt{apendice-ejemplo.tex} es una plantilla para añadir apéndices al \textsc{tfg}. Para ello, es necesario:
\begin{itemize}
  \item Crear una copia de este fichero \texttt{apendice-ejemplo.tex} en la carpeta \texttt{apendices} con un nombre apropiado (p.e. \texttt{apendice01.tex}).
  \item Añadir el comando \texttt{$\backslash$input\{apendices/apendice01\}} en el fichero principal \texttt{tfg.tex} donde queremos que aparezca dicho apéndice (debe de ser después del comando \texttt{$\backslash$appendix}).
\end{itemize}

\endinput
%------------------------------------------------------------------------------------
% FIN DEL APÉNDICE. 
%------------------------------------------------------------------------------------

% Añadir tantos apéndices como sea necesario 

% -------------------------------------------------------------------
% GLOSARIO: Opcional
% -------------------------------------------------------------------

% !TeX root = ../tfg.tex
% !TeX encoding = utf8

\chapter*{Glosario}
\addcontentsline{toc}{chapter}{Glosario} % Añade el glosario a la tabla de contenidos

La inclusión de un glosario es opcional.

Archivo: \texttt{glosario.tex}

\begin{description} 
  \item[$\mathbb{R}$] Conjunto de números reales.

  \item[$\mathbb{C}$] Conjunto de números complejos.

  \item[$\mathbb{Z}$] Conjunto de números enteros.
\end{description}
\endinput
 

% -------------------------------------------------------------------
% BACKMATTER
% -------------------------------------------------------------------

\backmatter % Desactiva la numeración de los capítulos
\pdfbookmark[-1]{Referencias}{BM-Referencias}

% BIBLIOGRAFÍA
%-------------------------------------------------------------------

\bibliographystyle{alpha-es} 
\begin{small} 
  \bibliography{library.bib}
\end{small}


\end{document}
