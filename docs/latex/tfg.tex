% !TEX program = pdflatex
% !TEX encoding = UTF-8 Unicode

% Plantilla, basada en la clase `scrbook` del paquete KOMA-script,  para la elaboración de un TFG siguiendo las directrices del la comisión del Grado en Matemáticas de la Universidad de Granada.

% Francisco Torralbo Torralbo

\documentclass[print, color]{ugrTFG}

% VERSIÓN ELECTRÓNICA PARA TABLETA
% Cambiando la opción "print" por "tablet" generaremos un pdf adaptado para leerlo en tabletas de 9 pulgadas.

% -------------------------------------------------------------------
% INFORMACIÓN DEL TFG Y EL AUTOR
% -------------------------------------------------------------------

\newcommand{\miTitulo}{Criptografía basada en retículos \xspace}
\newcommand{\miNombre}{Mario Rodríguez López\xspace}
\newcommand{\miGrado}{Doble Grado en Matemáticas e Ingeniería Informática}
\newcommand{\miFacultad}{Facultad de Ciencias y Escuela Técnica Superior de Ingenierías Informática y Telecomunicación}
\newcommand{\miUniversidad}{Universidad de Granada}

% Añadir tantos tutores como sea necesario separando cada uno de ellos mediante el comando `\medskip` y una línea en blanco
\newcommand{\miTutor}{
  Francisco Javier Lobillo Borrero \\ \emph{Departamento de Álgebra} 

  % Añadir tantos tutores como sea necesario. 

  \medskip
  Nombre del tutor 2 \\ \emph{Departamento del tutor 2}
}
\newcommand{\miCurso}{2024-2025\xspace}

\hypersetup{
	pdftitle={\miTitulo},
	pdfauthor={\textcopyright\ \miNombre, \miFacultad, \miUniversidad}
}

\begin{document}

\maketitle

% -------------------------------------------------------------------
% FRONTMATTER
% -------------------------------------------------------------------
\frontmatter % Desactiva la numeración de capítulos y usa numeración romana para las páginas

\input{preliminares/declaracion-originalidad}   
% !TeX root = ../tfg.tex
% !TeX encoding = utf8

%*******************************************************
% Dedication
%*******************************************************
\thispagestyle{empty}
\phantomsection 
\pdfbookmark[1]{Dedicatoria}{Dedicatoria}

\hfill
\vfill

\begin{flushright}
\itshape
A mis padres, por apoyarme siempre
\end{flushright}

\vfill

\cleardoublepage
\endinput
                % Opcional
\input{preliminares/tablacontenidos}            
\input{preliminares/agradecimientos}            % Opcional

\input{preliminares/summary}                    
\input{preliminares/introduccion}               

% -------------------------------------------------------------------
% MAINMATTER
% -------------------------------------------------------------------
\mainmatter % activa la numeración de capítulos, resetea la numeración de las páginas y usa números arábigos

% !TeX root = ../tfg.tex
% !TeX encoding = utf8

\chapter*{Motivación}
Vivimos en una era donde la digitalización ha transformado todos los aspectos de nuestra vida cotidiana. Desde la forma en que nos comunicamos, hasta cómo almacenamos información y accedemos a servicios esenciales, la dependencia de las tecnologías digitales es innegable. Esta digitalización, aunque ofrece innumerables beneficios en términos de eficiencia y conveniencia, también presenta desafíos significativos en cuanto a la protección de la información y la privacidad. Aquí es donde entra en juego la criptografía.\\

La criptografía es esencial en nuestra vida diaria, aunque muchas veces pase desapercibida. Su importancia radica en la capacidad de proteger la información confidencial, garantizar la privacidad y asegurar la integridad de los datos en un mundo cada vez más digitalizado. Desde el uso de tarjetas de crédito y transacciones bancarias en línea hasta la comunicación a través de aplicaciones de mensajería y el almacenamiento de datos personales en la nube, la criptografía asegura que estos procesos sean seguros y que la información no caiga en manos equivocadas.\\

Con el avance imparable de la tecnología, nos enfrentamos a una amenaza que desafía la seguridad de los métodos criptográficos tradicionales: la computación cuántica. \\

Actualmente, los algoritmos criptográficos más utilizados, como RSA y ECC (Criptografía de Curva Elíptica), se basan en problemas matemáticos complejos, como la factorización de números enteros grandes y el logaritmo discreto, que son extremadamente difíciles de resolver con la computación clásica. Sin embargo, estos problemas pueden ser resueltos de manera eficiente por los ordenadores cuánticos utilizando el algoritmo de Shor.\\

El algoritmo de Shor, desarrollado por el matemático Peter Shor en 1994, es capaz de factorizar números enteros grandes y resolver problemas de logaritmos discretos en un tiempo significativamente menor que los algoritmos clásicos. Esto implica que los ordenadores cuánticos podrían romper la mayoría de los sistemas criptográficos actuales, exponiendo información confidencial y comprometiendo la seguridad de los datos.\\

En este contexto, surge la necesidad urgente de desarrollar y adoptar sistemas criptográficos que sean resistentes a los ataques de la computación cuántica. Aquí es donde entra en juego el criptosistema post-cuántico Kyber-Crystals, una de las propuestas más prometedoras en el ámbito de la criptografía post-cuántica y el que fue seleccionado como el nuevo estándar de criptografía por el National Institute of Standards and Technology (NIST) en el año 2022, un reconocimiento que subraya su importancia y robustez frente a las amenazas cuánticas.\\

Kyber-Crystals se basa en problemas matemáticos en retículos o redes, que son considerados intratables incluso para los ordenadores cuánticos. Este enfoque garantiza que los datos cifrados bajo este sistema permanezcan seguros, resistiendo tanto a los ataques tradicionales como a los potenciales ataques cuánticos. La adopción de Kyber-Crystals como estándar de criptografía por el NIST es un hito significativo, ya que refleja una confianza institucional en su capacidad para proteger la información en un futuro dominado por la tecnología cuántica.


\endinput
%--------------------------------------------------------------------
% FIN DEL CAPÍTULO. 
%--------------------------------------------------------------------

\part{Fundamento Teórico. Reticulos \& MLWE} % Dividir un TFG en partes OPCIONAL
% !TeX root = ../tfg.tex
% !TeX encoding = utf8

\chapter{Preliminares}
Durante el desarrollo de este trabajo, se supondrá que el lector tiene conocimientos en álgebra, teoría de números y criptografía, además de una base en algoritmos y programación básica. En este capítulo se introducirán o recordarán algunos de los conceptos y definiciones necesarios para comprender el contenido de este trabajo.
\section{IND-CPA (Indistinguibilidad bajo ataques de texto cifrado elegido)}
Decimos que un sistema es seguro contra ataques de texto cifrado elegido si un atacante no puede distinguir dos mensajes cifrados elegidos por él mismo. En otras palabras, un sistema es seguro si un atacante no puede distinguir entre dos mensajes cifrados elegidos por él mismo.
\section{IND-CCA2 (Indistinguibilidad bajo ataques de texto cifrado elegido adaptativos)}
En criptografía, un esquema es IND-CCA2 si, incluso cuando un atacante puede elegir y descifrar cualquier cantidad de mensajes cifrados (excepto el que está tratando de romper), no puede descifrar el mensaje original que fue cifrado a menos que tenga la clave. 

\section{KEM (Key Encapsulation Mechanism)}
Un Key Encapsulation Mechanism (KEM) (Mecanismo de Encapsulación de Claves) es un esquema criptográfico que se utiliza para generar y compartir de forma segura una clave secreta entre dos partes. Es un componente esencial en muchos sistemas de criptografía híbrida, donde combina la seguridad del cifrado asimétrico con la eficiencia del cifrado simétrico.


\section{}
\section{}

\endinput
%--------------------------------------------------------------------
% FIN DEL CAPÍTULO. 
%--------------------------------------------------------------------
% !TeX root = ../tfg.tex
% !TeX encoding = utf8

\chapter{Teoría de Reticulos}

\section{Definición}

Un retículo en $\mathbb{R}^n$ es un subgrupo aditivo discreto de $\mathbb{R}^n$ definido como el conjunto de todas las combinaciones lineales enteras de n vectores linealmente independientes. Este conjunto de vectores se conoce como una base del retículo y no es único. Es decir un reticulo es el conjunto de todas las combinaciones lineales enteras de vectores de una base $B=\{b_1,b_2,\ldots,b_n\} \subset \mathbb{R}^n$.

\begin{equation}
    \mathcal{L} = \left\{ \sum_{i=1}^{n} a_i b_i \mid a_i \in \mathbb{Z} \right\}
\end{equation}


\endinput
%--------------------------------------------------------------------
% FIN DEL CAPÍTULO. 
%--------------------------------------------------------------------
\input{capitulos/capitulo03}
\input{capitulos/capitulo04}
\input{capitulos/capitulo05}
\input{capitulos/capitulo06}
\input{capitulos/capitulo07}
\input{capitulos/capitulo08}

% -------------------------------------------------------------------
% APPENDIX: Opcional
% -------------------------------------------------------------------

\appendix % Reinicia la numeración de los capítulos y usa letras para numerarlos
\pdfbookmark[-1]{Apéndices}{appendix} % Alternativamente podemos agrupar los apéndices con un nuevo \part{Apéndices}

\input{apendices/apendice-ejemplo}
% Añadir tantos apéndices como sea necesario 

% -------------------------------------------------------------------
% GLOSARIO: Opcional
% -------------------------------------------------------------------

\input{glosario} 

% -------------------------------------------------------------------
% BACKMATTER
% -------------------------------------------------------------------

\backmatter % Desactiva la numeración de los capítulos
\pdfbookmark[-1]{Referencias}{BM-Referencias}

% BIBLIOGRAFÍA
%-------------------------------------------------------------------

\bibliographystyle{alpha-es} 
\begin{small} 
  \bibliography{library.bib}
\end{small}


\end{document}
