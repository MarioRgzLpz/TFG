% !TeX root = ../tfg.tex
% !TeX encoding = utf8

\chapter*{Motivación}
Vivimos en una era donde la digitalización ha transformado todos los aspectos de nuestra vida cotidiana. Desde la forma en que nos comunicamos, hasta cómo almacenamos información y accedemos a servicios esenciales, la dependencia de las tecnologías digitales es innegable. Esta digitalización, aunque ofrece innumerables beneficios en términos de eficiencia y conveniencia, también presenta desafíos significativos en cuanto a la protección de la información y la privacidad. Aquí es donde entra en juego la criptografía.\\

La criptografía es esencial en nuestra vida diaria, aunque muchas veces pase desapercibida. Su importancia radica en la capacidad de proteger la información confidencial, garantizar la privacidad y asegurar la integridad de los datos en un mundo cada vez más digitalizado. Desde el uso de tarjetas de crédito y transacciones bancarias en línea hasta la comunicación a través de aplicaciones de mensajería y el almacenamiento de datos personales en la nube, la criptografía asegura que estos procesos sean seguros y que la información no caiga en manos equivocadas.\\

Con el avance imparable de la tecnología, nos enfrentamos a una amenaza que desafía la seguridad de los métodos criptográficos tradicionales: la computación cuántica. \\

Actualmente, los algoritmos criptográficos más utilizados, como RSA y ECC (Criptografía de Curva Elíptica), se basan en problemas matemáticos complejos, como la factorización de números enteros grandes y el logaritmo discreto, que son extremadamente difíciles de resolver con la computación clásica. Sin embargo, estos problemas pueden ser resueltos de manera eficiente por los ordenadores cuánticos utilizando el algoritmo de Shor.\\

El algoritmo de Shor, desarrollado por el matemático Peter Shor en 1994, es capaz de factorizar números enteros grandes y resolver problemas de logaritmos discretos en un tiempo significativamente menor que los algoritmos clásicos. Esto implica que los ordenadores cuánticos podrían romper la mayoría de los sistemas criptográficos actuales, exponiendo información confidencial y comprometiendo la seguridad de los datos.\\

En este contexto, surge la necesidad urgente de desarrollar y adoptar sistemas criptográficos que sean resistentes a los ataques de la computación cuántica. Aquí es donde entra en juego el criptosistema post-cuántico Kyber-Crystals, una de las propuestas más prometedoras en el ámbito de la criptografía post-cuántica y el que fue seleccionado como el nuevo estándar de criptografía por el National Institute of Standards and Technology (NIST) en el año 2022, un reconocimiento que subraya su importancia y robustez frente a las amenazas cuánticas.\\

Kyber-Crystals se basa en problemas matemáticos en retículos o redes, que son considerados intratables incluso para los ordenadores cuánticos. Este enfoque garantiza que los datos cifrados bajo este sistema permanezcan seguros, resistiendo tanto a los ataques tradicionales como a los potenciales ataques cuánticos. La adopción de Kyber-Crystals como estándar de criptografía por el NIST es un hito significativo, ya que refleja una confianza institucional en su capacidad para proteger la información en un futuro dominado por la tecnología cuántica.


\endinput
%--------------------------------------------------------------------
% FIN DEL CAPÍTULO. 
%--------------------------------------------------------------------
