% !TeX root = ../tfg.tex
% !TeX encoding = utf8

\chapter{Preliminares}
Durante el desarrollo de este trabajo, se supondrá que el lector tiene conocimientos en álgebra, teoría de números y criptografía, además de una base en algoritmos y programación básica. En este capítulo se introducirán o recordarán algunos de los conceptos y definiciones necesarios para comprender el contenido de este trabajo.
\section{IND-CPA (Indistinguibilidad bajo ataques de texto cifrado elegido)}
Decimos que un sistema es seguro contra ataques de texto cifrado elegido si un atacante no puede distinguir dos mensajes cifrados elegidos por él mismo. En otras palabras, un sistema es seguro si un atacante no puede distinguir entre dos mensajes cifrados elegidos por él mismo.
\section{IND-CCA2 (Indistinguibilidad bajo ataques de texto cifrado elegido adaptativos)}
En criptografía, un esquema es IND-CCA2 si, incluso cuando un atacante puede elegir y descifrar cualquier cantidad de mensajes cifrados (excepto el que está tratando de romper), no puede descifrar el mensaje original que fue cifrado a menos que tenga la clave. 

\section{KEM (Key Encapsulation Mechanism)}
Un Key Encapsulation Mechanism (KEM) (Mecanismo de Encapsulación de Claves) es un esquema criptográfico que se utiliza para generar y compartir de forma segura una clave secreta entre dos partes. Es un componente esencial en muchos sistemas de criptografía híbrida, donde combina la seguridad del cifrado asimétrico con la eficiencia del cifrado simétrico.


\section{}
\section{}

\endinput
%--------------------------------------------------------------------
% FIN DEL CAPÍTULO. 
%--------------------------------------------------------------------