% !TeX root = ../tfg.tex
% !TeX encoding = utf8

\chapter{Teoría de Reticulos}

\section{Definición}

Un retículo en $\mathbb{R}^n$ es un subgrupo aditivo discreto de $\mathbb{R}^n$ definido como el conjunto de todas las combinaciones lineales enteras de n vectores linealmente independientes. Este conjunto de vectores se conoce como una base del retículo y no es único. Es decir un reticulo es el conjunto de todas las combinaciones lineales enteras de vectores de una base $B=\{b_1,b_2,\ldots,b_n\} \subset \mathbb{R}^n$.

\begin{equation}
    \mathcal{L} = \left\{ \sum_{i=1}^{n} a_i b_i \mid a_i \in \mathbb{Z} \right\}
\end{equation}


\endinput
%--------------------------------------------------------------------
% FIN DEL CAPÍTULO. 
%--------------------------------------------------------------------